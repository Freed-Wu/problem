
%%%%%%%%%%%%%%%%%%%%%%%%%%%%%%%%%%%%%%%%%%%%%%%%%%%%%%%%%%%%%%%%
%
%  Template for homework of Introduction to Machine Learning.
%
%  Fill in your name, lecture number, lecture date and body
%  of homework as indicated below.
%
%%%%%%%%%%%%%%%%%%%%%%%%%%%%%%%%%%%%%%%%%%%%%%%%%%%%%%%%%%%%%%%%


\documentclass[11pt,letter,notitlepage]{article}
%Mise en page
\usepackage[left=2cm, right=2cm, lines=45, top=0.8in, bottom=0.7in]{geometry}
\usepackage{fancyhdr}
\usepackage{bm}
\usepackage{fancybox}
\usepackage{graphicx}
\usepackage{pdfpages}
\usepackage{enumitem}
\renewcommand{\headrulewidth}{1.5pt}
\renewcommand{\footrulewidth}{1.5pt}
\newcommand\Loadedframemethod{TikZ}
\usepackage[framemethod=\Loadedframemethod]{mdframed}

\usepackage{amssymb,amsmath}
\usepackage{amsthm}
\usepackage{thmtools}
\usepackage{mathrsfs}
\usepackage{afterpage}

\setlength{\topmargin}{0pt}
\setlength{\textheight}{9in}
\setlength{\headheight}{0pt}

\setlength{\oddsidemargin}{0.25in}
%\setlength{\evensidemargin}{0.25in}
\setlength{\textwidth}{6in}
\pagestyle{fancy}

%%%%%%%%%%%%%%%%%%%%%%%%
%%%%%% Define math operator %%%%%
%%%%%%%%%%%%%%%%%%%%%%%%
\DeclareMathOperator*{\argmin}{\bf argmin}
\DeclareMathOperator*{\relint}{\bf relint\,}
\DeclareMathOperator*{\dom}{\bf dom\,}
\DeclareMathOperator*{\intp}{\bf int\,}
%%%%%%%%%%%%%%%%%%%%%%%

%%%%%%%%%%%%%%%%%%%%%%%%
%% Define the Exercise environment %%
%%%%%%%%%%%%%%%%%%%%%%%%
\mdtheorem[
  topline=false,
  rightline=false,
  leftline=false,
  bottomline=false,
  leftmargin=-10,
  rightmargin=-10
]{exercise}{\textbf{Exercise}}
%%%%%%%%%%%%%%%%%%%%%%%
%% End of the Exercise environment %%
%%%%%%%%%%%%%%%%%%%%%%%


%%%%%%%%%%%%%%%%%%%%%%%%
%% Define the Problem environment %%
%%%%%%%%%%%%%%%%%%%%%%%%
\mdtheorem[
  topline=false,
  rightline=false,
  leftline=false,
  bottomline=false,
  leftmargin=-10,
  rightmargin=-10
]{problem}{\textbf{Problem}}
%%%%%%%%%%%%%%%%%%%%%%%
%% End of the Exercise environment %%
%%%%%%%%%%%%%%%%%%%%%%%

%%%%%%%%%%%%%%%%%%%%%%%
%% Define the Solution Environment %%
%%%%%%%%%%%%%%%%%%%%%%%
\declaretheoremstyle%
[
  spaceabove=0pt,
  spacebelow=0pt,
  headfont=\normalfont\bfseries,
  notefont=\mdseries,
  notebraces={(}{)},
  headpunct={:\quad},
  headindent={},
  postheadspace={ },
  postheadspace=4pt,
  bodyfont=\normalfont,
  qed=$\blacksquare$,
  preheadhook={\begin{mdframed}[style=myframedstyle]},
  postfoothook=\end{mdframed},
]{mystyle}

\declaretheorem[style=mystyle,title=Comment,numbered=no]{comment}
\mdfdefinestyle{myframedstyle}{%
  topline=false,
  rightline=false,
  leftline=false,
  bottomline=false,
  skipabove=-6ex,
  leftmargin=-10,
rightmargin=-10}

\declaretheorem[style=mystyle,title=Solution,numbered=no]{solution}
\mdfdefinestyle{myframedstyle}{%
  topline=false,
  rightline=false,
  leftline=false,
  bottomline=false,
  skipabove=-6ex,
  leftmargin=-10,
rightmargin=-10}
%%%%%%%%%%%%%%%%%%%%%%%
%% End of the Solution environment %%
%%%%%%%%%%%%%%%%%%%%%%%

%% Homework info.
\newcommand{\posted}{\text{Sep. 10, 2018}}       			%%% FILL IN POST DATE HERE
\newcommand{\due}{\text{Sep. 17, 2018}} 			%%% FILL IN Due DATE HERE
\newcommand{\hwno}{\text{3}} 		           			%%% FILL IN LECTURE NUMBER HERE

\newcommand{\rank}[1]{ \textbf{rank}  (#1)  }

%%%%%%%%%%%%%%%%%%%%
%% Put your information here %%
%%%%%%%%%%%%%%%%%%%
\newcommand{\name}{\text{}}  	          			%%% FILL IN YOUR NAME HERE
\newcommand{\id}{\text{}}		       			%%% FILL IN YOUR ID HERE
%%%%%%%%%%%%%%%%%%%%
%% End of the student's info %%
%%%%%%%%%%%%%%%%%%%

\lhead{
  \textbf{\name}
}
\rhead{
  \textbf{\id}
}
\chead{\textbf{
    Problem
}}


\begin{document}
\vspace*{-4\baselineskip}
\thispagestyle{empty}


\begin{center}
  {\bf\large Problem}\\
\end{center}

%\noindent
%Lecturer: Jie Wang  			 %%% FILL IN LECTURER HERE
%\hfill
%Homework \hwno
%\\
%Posted: \posted
%\hfill
%Due: \due
%\\
%Name: \hspace{100mm}
%\hfill
%ID: \id
%\hfill

\noindent
\rule{\textwidth}{2pt}

\medskip

Please select one out of the three problems to solve.

\begin{problem}[Second-order sufficient optimality conditions]
  Suppose that $f:\mathbb{R}^n\rightarrow\mathbb{R}$ is twice differentiable at $\bm{x}$. Please show that $\bm{x}$ is a strict local minimum if $\nabla f(\bm{x})=0$ and the Hessian matrix $\bm{H}(\bm{x})$ is positive definite.
\end{problem}

\begin{problem}[Low-rank approximation]
  Please find the solution to the problem as follows
  \begin{align*}
    \min_{X\in\mathbb{R}^{m\times n}}\{\|A-X\|_F:\rank{X}\leq K\},
  \end{align*}
  where $A\in \mathbb{R}^{m\times n}$ and $\rank{A}=r$.
\end{problem}

\begin{problem}[Random walk on $\mathbb{Z}$]
  Consider the random walk $ X=\{X_n\}_{n\geq 0} $ on $ \mathbb{Z} $ that starts at $ X_0=0 $. The particle moves with probability $ p $ one unit to the right and with probability $ q=1-p $ one unit to the left at each transition. Please show that the state $ 0 $ is recurrent if and only if $ p=q=1/2 $, i.e.,
  \begin{align*}
    \sum_{n=1}^{\infty} \mathbb{P}(X_n=0, X_k \neq 0, k = 1,\dots,n-1|X_0=0)=1\, \Leftrightarrow\, p=q=1/2.
  \end{align*}
\end{problem}

\begin{exercise}[Second-order sufficient optimality conditions]

  We have
  \begin{align}
    \nabla f(\bm{x}^0) = 0
  \end{align}

  and
  \begin{align}
    \bm{t}^\mathsf{T}\nabla^2f(\bm{\xi})\bm{t} > 0, \forall
    \bm{t} \neq \bm{0}
  \end{align}

  aka $\nabla^2f(\bm{\xi})$ is a positive-definite matrix.

  So,
  \begin{align}
    f(\bm{x}) = & f(\bm{x}^0) + \nabla{f(\bm{x})}^\mathsf{T}
    \Delta \bm{x} + \Delta \bm{x}^\mathsf{T}\nabla^2f(\bm{\xi})\Delta \bm{x}\\
    > & f(\bm{x}^0)
  \end{align}

  It means

  \begin{align}
    \exists\delta, \mathrm{s.t.}f(\bm{x}) > & f(\bm{x}^0), \forall \bm{x} \in U^\circ(\bm{x}^0, \delta)
  \end{align}

  QED.
\end{exercise}

\end{document}
